\clearpage
\pagenumbering{gobble}
\markboth{}{}

\makeatletter
\newlength\drop
\newcommand*{\titleGM}{\begingroup% Gentle Madness
	\drop = 0.08\textheight
	\vspace*{0.5\baselineskip}
	%\vfill
	\hbox{%
		\hspace*{0.2\textwidth}%
		\rule{1pt}{0.4\textheight}
		\hspace*{0.05\textwidth}%
		\parbox[b]{0.75\textwidth}{
			\vbox{%
				\vspace{\drop}
				{\hspace{-25pt}\raggedright\normalsize\@title\par
				}
				\vskip\baselineskip
				\centering
				{\@author%\par
				}
				\vskip16\baselineskip
				{\raggedright\@date\par}
				\vspace{0.04\textheight}
			}% end of vbox
		}% end of parbox
	}% end of hbox
	%\vfill
	\null
	\endgroup}
\makeatother


\title{ বইটির আংশিক আর্থিক দায়িত্ব নিয়ে দত্তক নিয়েছেন:}
\author{\large\textbf{\guruAdoptionList}}
\date{\scriptsize \textit{মানুষের বই পড়ার অভ্যাস নাকি হারিয়ে যাচ্ছে। অথচ বইয়ের দাম বেড়ে চলেছে লাফিয়ে লাফিয়ে। বোঝা যায়, ক্ষুদ্র একটা পাঠকবৃত্তকে সম্বল করে চলছে বই-ব্যাবসা। ছোট্টো পাঠকবৃত্ত, তাই বইয়ের দাম বেশি, আবার বইয়ের দাম বেশি বলে পাঠকসংখ্যা কম, তৈরি হচ্ছে এরকম এক অন্তহীন দুষ্টচক্র।\\
		এই চক্রব্যূহ থেকে বেরিয়ে আসার জন্যই আমাদের দাওয়াই ‘দত্তক’। গুরুচণ্ডা৯ লেখক-পাঠক সমবায়ে বিশ্বাসী, মুনাফায় নয়। গুরুচণ্ডা৯র শুভানুধ্যায়ীরা নিয়মিতভাবেই এক বা একাধিক বইয়ের সম্পূর্ণ বা আংশিক দায়ভার বহন করেন, যার পোশাকি নাম ‘দত্তক’। বাজারচলতি বইয়ের চেয়ে কমে আসে আমাদের বইয়ের দাম। পৌঁছে যায় আরও বহু মানুষের কাছে।\\
		আপনি কি এই মিথোজীবিতার সক্রিয় অংশীদার হতে চান? গুরুচণ্ডা৯-র বই দত্তক নিতে চান?\\
		যোগাযোগ করুন: guruchandali@gmail.com\\
		হোয়াটসঅ্যাপ: +৯১ ৯৩৩০৩ ০৮০৪৩}}
	
		\titleGM